\textbf{Danilo Paiva Ramos} é antropólogo e professor adjunto do Dept. de Antropologia e Etnologia da Universidade Federal da Bahia \versal{(UFBA)}. Tem mestrado, doutorado e pós-doutorado em Antropologia Social pela Universidade de São Paulo \versal{(USP)}, e pós-doutorado em Antropologia Linguística pela Universidade do Texas \versal{(UT)}. Além de suas pesquisas etnográficas, com ênfase em xamanismo, discurso, performance, vida ritual e territorialidade, é engajado em causas ligadas aos direitos indígenas. Paralelo à coordenação do Grupo de Estudos de Antropologia e Linguística \versal{(GEAL)} da \versal{USP}, participa do Coletivo de Apoio à Causa Yuhup-Hup \versal{(CAPYH)}, é assessor da Federação da Organizações Indígenas do Rio Negro \versal{(FOIRN)} em saúde indígena e da Funai em assuntos e projetos voltados aos povos Hupd'äh e Yuhupdëh. Também assessora o povo Hupd'äh para a construção de seus Planos de Gestão Territorial e Ambiental (\versal{PGTA-H}up).


\textbf{Círculos de coca e fumaça} debruça-se sobre os Hupd’äh, povo indígena falante de língua Hup que vive na região do Alto Rio Negro, no noroeste da Amazônia. Suas rodas noturnas para ingerir coca e tabaco -- momentos de compartilhar mitos e histórias de andanças pela mata, ensinar benzimentos e executar curas e proteções xamânicas -- são o principal cenário do livro. Nessas situações, Paiva Ramos percebeu performances, contextos em que os ameríndios relacionam suas experiências e observações da mata com as palavras dos mitos e encantamentos. A partir dessa interação, o viajante hup consegue interagir com seres de múltiplas paisagens e expandir seu campo de percepção, em um engajamento mútuo com os processos de transformação do mundo.


\textbf{Mundo Indígena}, coleção da Editora Hedra, reúne, de um lado, as cosmologias, histórias e reflexões de povos indígenas nas palavras de seus próprios pensadores, e, de outro, coletâneas e trabalhos acadêmicos de grandes estudiosos da questão indígena no Brasil, reafirmando assim sua existência e relevância em seus próprios termos.\par